\begin{tikzpicture}
    %There's a lot to learn about how tikz works. I recommend tinkering with this example to get a feel for how it works.
    \begin{scope}[every node/.style={circle,draw}]
        \node [label={$C_0$}](0) at (0,0) {};
        \node [label={$C_1$}](1) at (1.5,0) {};
        \node [label={$C_2$}](2) at (3,0) {};
        \node [label={$C_3$}](31) at (4.5,.5) {};
        \node (30) at (4.5,0) {};
        \node (32) at (4.5,-.5) {};
        \node [label={$C_4$}](41) at (6,1.25) {};
        \node (40) at (6,.5) {};
        \node (42) at (6,-.25) {};
        \node (43) at (6,-1) {};
        \node [label={$C_5$}] (51) at (7.5,1.25) {};
        \node (52) at (7.5,.5) {};
        \node (53) at (7.5,-.25) {};
        \node (54) at (7.5,-1) {};
        
        \node (b0) at (0,-2){};
        \node (b1) at (1.5,-2){};
        \node (b2) at (3,-2){};
        \node (b3) at (4.5,-2){};
        \node (b4) at (6,-2){};
        \node (b5) at (7.5,-2){};
    \end{scope}
    
    \begin{scope} [>={Stealth[black]},
                every node/.style={below right},
                every edge/.style={draw=black, thick}]
        \path [->] (1) edge (0);
        \path [->] (2) edge (1);
        \path [->] (31) edge (2);
        \path [->] (30) edge (2);
        \path [->] (32) edge (2);
        \path [->] (41) edge (31);
        \path [->] (40) edge (30);
        \path [->] (42) edge (32);
        \path [->] (43) edge (32);
        \path [->] (51) edge (41);
        \path [->] (52) edge (40);
        \path [->] (53) edge (42);
        \path [->] (54) edge (43);
        
        \path [->] (b5) edge node {$f$} (b4);
        \path [->] (b4) edge node {$f$} (b3);
        \path [->] (b3) edge node {$f$} (b2);
        \path [->] (b2) edge node {$f$} (b1);
        \path [->] (b1) edge node {$f$} (b0);
    \end{scope}
    
    \begin{scope}[>={Stealth[black]},
                every node/.style={left},
                every edge/.style={draw=black,  thick}]
        \path [->] (0) edge node {$u_0$} (b0);
        \path [->] (1) edge node {$u_1$} (b1);
        \path [->] (2) edge node {$u_2$} (b2);
        \path [->] (32) edge node {$u_3$} (b3);
        \path [->] (43) edge node {$u_4$} (b4);
        \path [->] (54) edge node {$u_5$} (b5);
    \end{scope}
    
\end{tikzpicture}